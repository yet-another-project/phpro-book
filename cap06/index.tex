\chapter{Programare orientată pe obiecte}
\section{Principii OOP de bază}
\textsl{Programarea orientată pe obiecte} (en. \textsl{object-oriented programming}, abv. \textsl{OOP})
ne permite simularea realităţii folosind aşa-numitele obiecte.

Aceste obiecte au \textsl{proprietăţi} şi \textsl{metode}. O pisică de exemplu poate avea
proprietăţile \textit{nume}, \textit{vârstă}, \textit{culoare}, \textit{greutate}.

O metodă a unui obiect reprezintă o acţiune pe care o poate face acel obiect. O pisică
poate mieuna, mânca, dormi sau se uita pe fereastră. Pentru fiecare acţiune,
pisica noastră va avea o metodă.

PHP poate crea un obiect pe baza unui \textit{plan de construcţie} numit \textsl{clasă}.
Deci înainte de a crea un obiect, trebuie să îi definim acest \textit{plan de construcţie}.
Spunem că \textit{implementăm o clasă}, la fel cum spuneam şi despre funcţii
că le-am implementat.

Asta se face folosind cuvântul cheie \texttt{class}, iar în interiorul unui bloc înconjurat
de acolade (la fel ca o funcţie) scriem definiţia acelei clase.

Haideţi să definim un posibil plan de construcţie al unei pisici:

\begin{lstlisting}[title=An empty class]
<?php
class Cat {

}
$mycat = new Cat;
$hiscat = new Cat;
\end{lstlisting}

Liniile 2-4 definesc o nouă clasă numită \texttt{Cat}, iar această clasă este goală.
Pe liniile 5 şi 6 creăm două obiecte de tip \texttt{Cat} şi salvăm aceste obiecte
în variabilele \texttt{\$mycat} şi respectiv \texttt{\$hiscat}.

Crearea de obiecte se face folosind cuvântul cheie \texttt{new} urmat de numele clasei.
Mai spunem şi că am \textsl{instanţiat} clasa \texttt{Cat}.

Acest concept al instanţierii se potriveşte cu ceea ce facem în viaţa reală:
de îndată ce ai planul de construcţie al unei case, poţi crea oricâte case
absolut identice. Toate casele vor avea aceleaşi facilităţi şi aceleaşi
proprietăţi cum sunt stabilite de arhitect în acel \textit{plan de construcţie}
de pe hârtie.

Haideţi să ne uităm mai întâi puţin la cum adăugăm proprietăţi unui plan de construcţie
şi cum folosim aceste proprietăţi după ce am creat instanţe concrete după acel plan.

\begin{lstlisting}[title=Public Class Properties]
<?php
class Cat {
  public $name;
  public $age;
  public $colour;
}

$mycat = new Cat;
\end{lstlisting}

Acum am adăugat trei proprietăţi planului de construcţie al clasei \texttt{Cat}. Toate
instanţele acestei clase vor avea aceste proprietăţi. Nu vor exista pisici fără un
nume, o vârstă, şi o culoare.

Aceste proprietăţi sunt \textsl{publice}. Vom reveni mai târziu asupra acestui aspect,
însă ţine minte: \texttt{public} este un specificator de acces (en. \textsl{access specifier})
care ne permite accesarea din exteriorul instanţelor a proprietăţilor sau metodelor pe 
care le califică.

Haideţi să vedem ce înseamnă asta în cod, concret:
\begin{lstlisting}[firstnumber=9,title={Public Class Properties}]
$mycat->name = 'Cherry';
$mycat->age = 3;
$mycat->colour = 'black';

echo 'This is '.$mycat->name.', my kitty, it\'s '.$mycat->age.
  ' years old, and as you can see it\'s '.$mycat->colour.'.<br />';

$hiscat = new Cat;
echo 'His cat is called "',$hiscat->name,'"';
\end{lstlisting}

Cu operatorul \texttt{->} (en. \textsl{object operator}) accesăm lucrurile din interiorul unei instanţe.
Dacă acele proprietăţi nu ar fi fost publice, nu le-am fi putut accesa din exterior
cu acest operator.
%TODO ask about replacing "lucrurile" with "proprietăţile şi metodele"

Pe linia 16 creăm o nouă instanţă a acestei clase. După cum vezi, cele două pisici sunt complet
separate, fiecare cu proprietăţile sale. De aceea, şi pentru
că nu am iniţializat nicăieri \texttt{\$hiscat->name}, aceasta este \texttt{NULL}.

\attention{Fiecare instanţă a unei clase este independentă de celelalte instanţe, fie
ele şi din aceeaşi clasă.}

Haideţi să vedem ce se întâmplă atunci când o proprietate nu este publică.
Schimbă accesul proprietăţii \texttt{age} din \texttt{public}
în \texttt{private}. Vei vedea că nu mai poţi accesa acea proprietate din exteriorul planului
de construcţie (al clasei).

De fapt, este şi de dorit ca proprietăţi ca \texttt{age} să fie privată. După cum vezi,
aceste proprietăţi pot fi suprascrise cu orice fel de informaţii. Însă fiecare proprietate
are specificul său. Proprietatea \texttt{age} de exemplu trebuie să fie mereu un număr pozitiv.

Dar cum setăm \texttt{age}, dacă proprietatea rămâne privată? Aici intervine al doilea lucru
pe care îl poate conţine un astfel de \textit{plan de construcţie}: \textsl{metodele}.

O metodă nu este nimic altceva decât o funcţie. Însă implementarea acestei funcţii se află
în interiorul definiţiei clasei, aşa cum era de aşteptat.

Deci haideţi să introducem două metode noi pentru citirea din şi scrierea în proprietatea
\texttt{age} a fiecărui obiect de tip \texttt{Cat}.

\begin{lstlisting}[title=Getters and Setters]
<?php
class Cat {
  private $name;
  private $age;
  private $colour;

  public function getAge() {
    return $this->age;
  }

  public function setAge($age) {
    if(is_numeric($age)) {
      $this->age = $age;
      return TRUE;
    }
    return FALSE;
  }
}

$mycat = new Cat;
var_dump($mycat->setAge('hello'));
var_dump($mycat->setAge(3));

echo 'My kitty is ',$mycat->getAge(),' years old.';
\end{lstlisting}

În interiorul unei clase avem acces la variabila \texttt{\$this} pe care
după cum vezi o şi folosim în interiorul metodelor. Această variabilă
este creată atunci când fluxul de execuţie ajunge pe linia 20, unde instanţiem
o clasă cu operatorul \texttt{new}. \texttt{\$this} face referire la acelaşi obiect la care face
referire şi \texttt{\$mycat} în exteriorul clasei.

În interiorul clasei însă nu ştim şi nu ne pasă cum este numită variabila
în care salvăm această instanţă în exterior (aici: \texttt{\$mycat}).
În \texttt{\$this} avem instanţa curentă,
şi prin ea, avem acces la toate metodele şi proprietăţile instanţei.

Deoarece proprietatea \texttt{age} este privată, nu o putem accesa direct,
fie că vrem să salvăm ceva în ea, fie că vrem să citim din ea.

Din acest motiv, pe linia 21 apelăm metoda responsabilă pentru salvarea
de informaţii în \texttt{age}. Atunci când apelăm o metodă a unei instanţe,
mai spunem şi că îi trimitem un mesaj acelei instanţe/obiect.

Deşi obiectul primeşte acest mesaj pe care i-l transmitem, proprietatea
\texttt{age} nu este setată deoarece planul de construcţie (implementarea clasei)
este în aşa fel făcut încât nu acceptă decât numere.

În acest fel ne asigurăm că informaţiile salvate în obiect sunt mereu valide,
indiferent de ce mesaje primim din exterior. Mai spunem şi că astfel
impunem integritatea datelor.

Pe linia 24 apelăm metoda obiectului pentru citirea proprietăţii \texttt{age}.

După cum ţi-ai dat seama, pentru fiecare proprietate pe care vrem să
o facem accesibilă cumva din exterior, dar în acelaşi timp să îi garatăm
integritatea, trebuie să creăm o pereche de astfel de două metode: una
pentru a salva date, alta pentru a primi date de la obiect.

La modul general, o metodă care setează valoarea unei proprietăţi se numeşte
\textsl{setter}, iar o metodă care returnează valoarea unei proprietăţi se
numeşte \textsl{getter}.

Bineînţeles că nu este musai ca o clasă să aibă câte o pereche
de getteri şi setteri pentru fiecare proprietate.
E foarte posibil de exemplu să creăm o clasă care are doar setter pentru
o anumită proprietate, iar această clasă va funcţiona ca un sac
fără fund, deoarece nu putem prelua informaţia din instanţe
de acel tip, ci doar {\glqq}pune în sac{\grqq}. Sau invers, doar
un getter, fără setter. Sau chiar o proprietate care nu este în
niciun fel expusă exteriorului.

\begin{Exercise}[title={A complete Kitty}]
\ExePart
Completează implementaţia clasei \texttt{Cat} cu getteri şi setteri
pentru celelalte două proprietăţi. Setterii trebuie să valideze
mesajele primite: un nume valid este constituit din doar un cuvânt
de lungime maximă 32 (de caractere), iar culoarea trebuie
să fie una din opt culori valide.

\ExePart
Adaugă o metodă care returnează o listă a tuturor culorilor pe
care le-a avut pisica de-a lungul timpului.
\end{Exercise}


Retrospectiv putem spune: o clasă este definiţia unui nou tip de date
care înglobează proprietăţi (informaţii salvate în variabile) şi
metode (cod executabil sub formă de funcţii) ce operează pe aceste
proprietăţi.

%TODO constructor
%TODO self-containment: o clasă nu poate accesa proprietăţile bicicletei adăpostite în ea, îi poate doar trimite mesaje





% \section{AJAX. Studiu de caz: guestbook}
% introducere in javascript, firebug
% 
% jquery
% 
% \section{Temă: crearea unui blog}
% blog
% \section{Temă: crearea unui forum}
% Crearea unui forum
% 
