\begin{Exercise}[title={A taste of git}]
Dacă ești în programul de tutelare, ai deja cont pe github.com.
Acest site îți oferă hosting pentru proiectele tale. Intră
în contact cu noi pe IRC pentru a fi îndrumat.

Crează un nou proiect pe github.com denumit la fel ca username-ul
tău de pe github.com, iar apoi clonează-l local și experimentează
cu el.

Pentru lucrul cu repozitoriile hostate pe github, citește instrucțiunile
de configurare de pe github\footnote{\url{http://help.github.com/}}.

Întreabă-i pe ceilalți cursanți ce lucruri interesante mai poți face
cu git.

Încearcă să te joci cât mai mult cu acest utilitar, deoarece de acum
încolo toate proiectele tale vor fi puse sub revision control în git.

Îți amintești probabil și de "gist-ul" unde ți-ai salvat toate soluțiile
la exerciții până acum. Ei bine, un "gist" nu este nimic altceva decât
un repozitoriu git. Îl poți clona bine mersi, și opera pe el
din confortul consolei tale -- nu e nevoie să accesezi interfața
web pentru a posta ceva.

Crează un fork al cărții, fă mici corecturi acolo unde vezi necesar,
dacă ai găsit greșeli în carte, învață să faci pull
requests, să trimiți patchuri, să deschizi issues pe interfața web
a proiectelor tale pusă la dispoziție de github.

Investește în jur de 40-80 de ore de lectură și de experimente pentru
a deveni cât de cât rutinat în folosirea acestor sisteme: git
ca utilitar de sine stătător pe de o parte, și github.com ca
hosting pentru proiectele tale.
\end{Exercise}
